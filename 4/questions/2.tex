\section{تشخيص پذيری و تصميم پذيری}
\subsection{}
در ابتدا ثابت می‌کنیم اگر
$D$
تصمیم پذیر باشد آنگاه
$C = \{x | \exists y ~ \langle x, y \rangle \in D\}$
تشخیص پذیر است. برای این موضوع کافی است که تمام رشته‌های ممکن برای
$y$
را تست کنیم. این ماشین ممکن است که تا ابد ادامه پیدا کند چرا که ممکن است هیچ
$y$ای وجود نداشته باشد.

حال برای حالت برعکس حساب می‌کنیم. یعنی فرض کنید که یک زبان
$C$
می‌دانیم که تشخیص پذیر است. باید یک
\lr{subset}
از آن را پیدا کنیم که تصمیم پذیر باشند. برای این کار صرفا کافیست که
$y$
را برابر تعداد مراحلی که باید ماشین را اجرا کنیم تا
\lr{halt}
رخ دهد قرار می‌دهیم. در صورتی که ماشین
\lr{halt}
نکند هیچ
$y$ای
وجود ندارد پس رشته قبول نمی‌شود. از آنجا که تعداد عملیات نیز
\lr{bounded}
است پس قطعا ماشین
\lr{D}
\lr{halt}
می‌کند.

\subsection{}
از طرف
$L$
به
$L^*$
کافی است که بدین صورت عمل کنیم که در ابتدا چک کنیم که آیا ورودی
$\epsilon$
است یا خیر، در صورتی که آن بود، رشته را قبول کنیم. سپس رشته را به تکه‌هایی به طول‌‌های متفاوت بشکنیم
و همزمان با
\lr{scheduling}
آنها را به ماشین تورینگ بدهیم که بر روی خودش شبیه سازی بکند.
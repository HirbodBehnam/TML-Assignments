\section{زبان‌های نامظم و لم تزریق}
\subsection{}
\subsubsection{}
این زبان منظم است. دقت کنید که جفت اعدادی که درست می شوند طول یکسانی دارند. برای همین می توانیم از
پرارزش ترین رقم هر دو عدد شروع کنیم و به سمت کم ارزش ترین رقم حرکت کنیم که در اینجا این کار
برابر خواندن رشته است. زمانی که دو رقم عدد بالا و پایین برابر بودند کاری نمی‌کنیم. اما به محض اینکه یک
رقم متفاوت دیدیم بزرگی یا کوچکی آن مشخص می‌شود. با تمام این حرف‌ها
\lr{DFA}
این زبان به صورت زیر است:
\begin{latin}
    \centering
    \begin{tikzpicture}[->, >=stealth', auto, semithick, node distance=2cm]
    \tikzstyle{every state}=[fill=white,draw=black,thick,text=black,scale=1]
    \node[initial,state,initial text=] (eq)  {$q_{eq}$};
    \node[state,accepting] (gt) [above right of=1] {$q_{gt}$};
    \node[state] (lt) [below right of=1] {$q_{lt}$};
    \path
    (eq) edge[loop above, above] node{$\binom{0}{0}, \binom{1}{1}$} (eq)
    (eq) edge[above] node{$\binom{1}{0}$} (gt)
    (eq) edge[below] node{$\binom{0}{1}$} (lt)
    (gt) edge[loop above, above] node{$\Sigma$} (gt)
    (lt) edge[loop below, below] node{$\Sigma$} (lt)
    ;
    \end{tikzpicture}
\end{latin}
\subsubsection{}
این زبان نامنظم است و به کمک
\lr{pumping lemma}
این را نشان می‌دهیم. فرض کنید که
$p$
برابر
\lr{pumping length}
باشد. آنگاه
$\binom{1}{0}^p \binom{0}{1}^p \in L_2$
است. با توجه به
\lr{pumping lemma}
می‌دانیم که این رشته را می‌توان به سه قسمت
$xyz$
شکاند به طوری که
$|xy| \le p$
است. با توجه به طول رشته‌ای که ما در نظر گرفتیم
$y = \binom{1}{0}^\alpha$
است که
$\alpha \le p$
است. حال طبق این لم داریم که مثلا
$\binom{1}{0}^{p + n} \binom{0}{1}^p$
نیز باید عوض
$L_2$
باشد به طوری که
$n > 0$
باشد. اما به وضوح مشخص است که اصلا تعداد صفر یک‌های دو ردیف در حال حاضر فرق دارند پس این رشته عوض
$L_2$
نیست. پس در نتیجه این زبان منظم نیست.
\subsection{}
فرض کنید که
\lr{pumping length}
برابر
$p$
باشد. حال رشته‌ی
$(^p)^p$
را در نظر بگیرید. این رشته عضو زبان ما است. اما دقیقا با همان استدلال سوال قبل رشته‌ی
$(^{p+n})^p$
که
$n > 0$
است عضو این زبان نیست.
\subsection{}
می‌دانیم که زبان‌های منظم نسبت به مکمل، اشتراک، اجتماع،
\lr{concat} و \lr{star}
بسته هستند. پس اگر مثلا دو زبان را با هم اشتراک بگیریم و نتیجه یک زبان نامنظم شود نشان می‌دهد که
یکی از دو زبان نیز نامظم بودند.
\subsubsection{}
\begin{gather*}
    L_2 \cap (1^* 0^*) = \{1^y 0^y | y \ge 0\}
\end{gather*}
از آنجا که سمت راست تساوی یک زبان نامنظم شد که در صورت سوال ذکر شده بود، پس سمت چپ نیز نمی‌تواند جفتشان
منظم باشد. مشخص است
$(1^* 0^*)$
منظم است. پس نتیجه می‌شود که
$L_2$
نامنظم است.
\subsubsection{}
\begin{gather*}
    L_3 \cap (0 1^* 2^*) = 0 \circ \{1^x 2^x | x \ge 0\}
\end{gather*}
در اینجا سمت راست تساوی نامظم است چرا که یک رشته‌ی ثابت با یک الفبای کاملا متفاوت به زبان داخل صورت سوال
\lr{concat}
شده است. پس در نتیجه سمت چپ نیز باید نامظم باشد. دقت کنید که 
$(0 1^* 2^*)$
منظم است. پس باید
$L_3$
نامظم باشد.
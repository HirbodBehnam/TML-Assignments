\section{عبارت‌های منظم}
\subsection{}
راهی که برای حل کردن این سوال در پیش گرفتم این بود که سعی کنم که عبارتی طراحی کنم اجتماع رشته‌هایی با عمق صفر
تا چهار باشند. در ابتدا مشخص است که
$\epsilon$
یک رشته‌ی قابل قبول است. سپس سراغ رشته‌هایی با عمق یک می‌رویم.
این رشته‌ها عملا از چندین پرانتز باز و بسته دقیقا پیشت سر هم تشکیل شده‌اند. مثلا
$()()()$.
عبارت منظم این رشته‌ها عملا برابر است با
$\{()\}^*$
پس عملا تا اینجا زبان ما برابر است با
$\epsilon \cup \{()\}^*$.
اما نکته‌ای که وجود دارد این است که
$\epsilon \in \{()\}^*$
است و عملا زبان ما تا اینجا صرفا خود
$\{()\}^*$
است. حال سعی ‌می‌کنیم که رشته‌ها با عمق دو را نشان دهیم. این رشته‌ها عبارتند از
$\{\left({\{()\}^*}\right)\}^*$
اما با کمی توجه متوجه می‌شویم که این زبان رشته‌هایی با طول دو را نشان نمی‌دهد. بلکه رشته‌هایی با عمق حداکثر
دو را نشان می‌دهد! به عنوان مثال دقت کنید که در اینجا اپسیلون نیز عضو این عبارت است. همچنین در صورتی که
عملیات
\lr{star}
درونی را برابر اپسیلون قرار دهیم عملا باز به
$\{()\}^*$
می‌رسیم!
حال سعی می‌کنیم که رشته‌هایی با عمق ۳ یا کمتر را نشان دهیم. مشخص است که باید از رشته‌ی قبلی استفاده کنیم.
پس داریم:
\begin{gather*}
    \{\left({
        \{\left({
            \{()\}^*
        }\right)\}^*
    }\right)\}^*
\end{gather*}
و برای رشته‌هایی با حداکثر عمق چهار داریم:
\begin{gather*}
    \{\left({
        \{\left({
            \{\left({
                \{()\}^*
            }\right)\}^*
        }\right)\}^*
    }\right)\}^*
\end{gather*}
\subsection{}
عملا این تابع کاری که می‌کند این است که یک رشته را می‌گیرد و هر حرف آنرا دوبار تکرار می‌کند.
به عنوان مثال رشته‌ی
$aba$
تبدیل به
$aabbaa$
می‌شود. سوال از ما خواسته است که عملا رشته‌هایی را در یک زبان پیدا کنیم که دو بار هر یک از حروف آن تکرار شده‌اند.

فرض کنید که زبان
$L$
\lr{DFA}اش
برابر شکل زیر است:
\begin{latin}
    \centering
    \begin{tikzpicture}[->, >=stealth', auto, semithick, node distance=2cm]
    \tikzstyle{every state}=[fill=white,draw=black,thick,text=black,scale=1]
    \node[initial,state,initial text=] (1)  {$q_1$};
    \node[state] (2)[right of=1] {$q_2$};
    \node[state,accepting] (3)[right of=2] {$q_3$};
    \node[state] (4)[below of=1] {$q_4$};
    \node[state,accepting] (5)[below of=4] {$q_5$};
    \node[state] (6)[right of=4] {$q_6$};
    \path
    (1) edge[below] node{$a$} (2)
    (2) edge[below] node{$a$} (3)
    (1) edge[left] node{$b$} (4)
    (4) edge[below] node{$b$} (6)
    (4) edge[left] node{$a$} (5)
    ;
    \end{tikzpicture}
\end{latin}
در این صورت
$slammer^{-1}(L)$
برابر است با
\lr{DFA}
زیر:
\begin{latin}
    \centering
    \begin{tikzpicture}[->, >=stealth', auto, semithick, node distance=2cm]
    \tikzstyle{every state}=[fill=white,draw=black,thick,text=black,scale=1]
    \node[initial,state,initial text=] (1)  {$q_1$};
    \node[state] (2)[right of=1] {$q_2$};
    \node[state,accepting] (3)[right of=2] {$q_3$};
    \node[state] (4)[below of=1] {$q_4$};
    \node[state,accepting] (5)[below of=4] {$q_5$};
    \node[state] (6)[right of=4] {$q_6$};
    \path
    (1) edge[bend left, above] node{$a$} (3)
    (1) edge[left] node{$b$} (6)
    ;
    \end{tikzpicture}
\end{latin}
عملا باید از روی
\lr{transition}های
تکراری به فاصله‌ی دو بپریم بر روی حالت بعدی. به عبارت دیگر
$\delta'(q, a) = \delta(\delta(q, a), a)$
است.
\subsection{}
\subsubsection{}
در صورتی که قرار دهیم
$xy^*z = a$
و
$xyz = b$
آنگاه عملا صورت سوال تبدیل می‌شود به:
\begin{gather*}
    a^*(ba^*)^*
\end{gather*}
که عملا یکی از عبارات قسمت
14
می‌شود. حال می‌دانیم که با توجه به اینکه تمامی این عبارات با هم برابر اند داریم که صورت سوال برابر است با
\begin{gather*}
    (xyz \cup xy^*z)^*
\end{gather*}
حال طبق مساوی‌های دیگر داریم:
\begin{align*}
    (xyz \cup xy^*z)^* &= (x(yz \cup y^*z))^* && (7)\\
    &= (x(y \cup y^*)z)^* && (8)
\end{align*}
حال ثابت می‌کنیم که
$x \cup x^* = x^*$
است. برای این موضوع صرفا از تعریف استفاده می‌کنیم که
$x \subseteq x^*$
چرا که
$x^*$
یکی از حالاتش همان
$x$
است. پس در اثبات اصلی ادامه می‌دهیم که:
\begin{gather*}
    (x(y \cup y^*)z)^* = (xy^*z)^*
\end{gather*}
\subsubsection{}
باید ثابت کنیم که هر رشته‌ای از هر طرف مساوی انتخاب کنیم زیر مجموعه‌ی طرف دیگر است. در صورتی که
$A \subseteq B$ و $B \subseteq A$
باشد، آنگاه می‌توان نتیجه گرفت که
$A = B$
است.
$(x \cup y)^*$
عملا تمامی ترکیب‌های حاوی
$x$ و $y$
را بدست می‌آورد. پس قطعا
$(x^*y)^*x^*$ و $x^*(yx^*)^*$
زیر مجموعه‌ی
$(x \cup y)^*$
هستند. حال برای اثبات از آن طرف، کاری که می‌کنیم این است که در ابتدا توجه می‌کنیم که
$(x^*y)^*$
عملا رشته‌هایی را می‌دهد که در ابتدای آن چندین
$x$
وجود داشته باشد و با
$y$
تمام می‌شود. دقت کنید که تعداد
$x$ها
می‌تواند 0 باشد. پس در نتیجه با کمک
$(x^*y)^*$
می‌توان تمام رشته‌هایی را ساخت که با
$y$
تمام شوند و یا اینکه
$\epsilon$
باشند. اما نمی‌توان با این رشته، رشته‌هایی را ساخت که با
$x$
تمام می‌شوند و یا
$y$
به صورت کلی نداشته باشند. برای درست کردن این موضوع می‌توان به آخر این عبارت
$x^*$
را چسباند. با این کار در ابتدا رشته را به رشته‌ای می‌شکانیم به دو قسمت می‌شکانیم که آخر قسمت اول باید
با
$y$
تمام شود. سپس بقیه‌ی
$x$ها
را به کمک
$x^*$
می‌سازیم در آخر رشته. پس از طرفی دیگر
$(x \cup y)^* \subseteq (x^*y)^*x^*$
است در نتیجه
$(x \cup y)^* = (x^*y)^*x^*$
است. از طرف دیگر برای
$x^*(yx^*)^*$
به صورت مشابه می‌توان استدلال مشابه‌ای کرد. بدین صورت که رشته را به قسمتی می‌شکانیم که قسمت اول آن فقط
$x$
داشته باشد که توسط
$x^*$
ساخته می‌شود. در ادامه‌ی آن نیز با استدلال مشابه به کمک
$(yx^*)^*$
می‌توان رشته‌هایی را ساخت که اول آنها با
$y$
شروع می‌شود. پس عملا
$x^*(yx^*)^* = (x \cup y)^*$
است.
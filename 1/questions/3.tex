\section{خواس بستاری زبان‌های منظم}
\subsection{}
مانند اثبات بسته بودن بر روی اشتراک و اجتماع عمل می‌کنیم. فرض کنیم که
$M_1 = (Q_1, \Sigma, \delta_1, q_1, F_1)$
و
$M_2 = (Q_2, \Sigma, \delta_2, q_2, F_2)$
است. ماشین
$M_3$
را به صورت زیر تعریف می‌کنیم.
\begin{gather*}
    Q = Q_1 \times Q_2\\
    \Sigma = \Sigma\\
    \delta((q, r), a) = (\delta_1(q, a), \delta_2(q, a))\\
    q_0 = (q_1, q_2)\\
    F = (F_1 \times Q_2) \cup (Q_1 \times F_2) - (F_1 \times F_2)
\end{gather*}
عملا عملگر داده شده در سوال همان دلتا
($\Delta$)
است که برابر است با
$L_1 \cup L_2 - L_1 \cap L_2$
\subsection{}
به زبان ساده‌تر این عملگر برابر است با تمام رشته‌هایی از زبان که یک حرف آن‌ها حذف شده است. برای طراحی
این ماشین از
\lr{NFA}
استفاده استفاده می‌کنیم. در ابتدا به صورت شهودی توضیح می‌دهم که چه کار باید بکنیم.  به صورت شهودی باید
دقیقا یک ماشین دیگر با همان
\lr{transition}های زبان
$L_3$
را داشته باشیم. سپس حالات
\lr{accepting state}
ماشینی که از آن شروع می‌کنیم را بر می‌داریم. سپس برای هر حالت از ماشین اول، با یک
\lr{eplislon transition}
به ماشین دوم می‌رویم اگر بین دو حالت در ماشین اصلی یک مسیر وجود داشته باشد. دقت کنید که این موضوع بدین
معنا است که در صورتی که
\lr{loop}
داشته باشم، یک
\lr{eplislon transition}
به همان
\lr{state}
در ماشین دوم داریم. به عنوان مثال فرض کنید که
$L_3$
به کمک ماشین زیر شناخته می‌شود:
\begin{latin}
    \centering
    \begin{tikzpicture}[->, >=stealth', auto, semithick, node distance=3cm]
    \tikzstyle{every state}=[fill=white,draw=black,thick,text=black,scale=1]
    \node[initial,state,initial text=] (1)  {$q_1$};
    \node[state] (2)[right of=1] {$q_2$};
    \node[state] (3)[right of=2] {$q_3$};
    \node[state,accepting] (4)[right of=3] {$q_4$};
    \path
    (1) edge[below] node{$a$} (2)
    (2) edge[below] node{$b$} (3)
    (3) edge[below] node{$c$} (4)
    (2) edge[loop above] node{$a$} (2)
    ;
    \end{tikzpicture}
\end{latin}
در صورتی که بخواهیم عملیات داده شده بر روی سوال را بر روی این
\lr{DFA}
انجام دهیم به
\lr{NFA}
زیر می‌‌رسیم:
\begin{latin}
    \centering
    \begin{tikzpicture}[->, >=stealth', auto, semithick, node distance=3cm]
    \tikzstyle{every state}=[fill=white,draw=black,thick,text=black,scale=1]
    \node[initial,state,initial text=] (1)  {$q_1$};
    \node[state] (2)[right of=1] {$q_2$};
    \node[state] (3)[right of=2] {$q_3$};
    \node[state] (4)[right of=3] {$q_4$};
    \node[state] (5)[below of=1] {$q_1'$};
    \node[state] (6)[below of=2] {$q_2'$};
    \node[state] (7)[below of=3] {$q_3'$};
    \node[state,accepting] (8)[below of=4] {$q_4'$};
    \path
    (1) edge[below] node{$a$} (2)
    (2) edge[below] node{$b$} (3)
    (3) edge[below] node{$c$} (4)
    (2) edge[loop above] node{$a$} (2)
    (5) edge[below] node{$a$} (6)
    (6) edge[below] node{$b$} (7)
    (7) edge[below] node{$c$} (8)
    (6) edge[loop below] node{$a$} (6)
    (1) edge[above] node{$\epsilon$} (6)
    (2) edge[above] node{$\epsilon$} (7)
    (3) edge[above] node{$\epsilon$} (8)
    (2) edge[left] node{$\epsilon$} (6)
    ;
    \end{tikzpicture}
\end{latin}
حال سعی می‌کنیم که به زبان ریاضی ماشین نهایی را تعریف کنیم. برای این کار همان طور که گفته شد نیاز
به دو برابر حالات ماشین اول داریم. برای این کار من از
$Q = \{0,1\} \times Q_L$
استفاده می‌کنم.
$\Sigma$
که تغییر نمی‌کند.
$q_0 = (0, q_{0_L})$
است و
$F$
برابر است با
$\{1\} \times F_L$.
حال برای تعریف
$\delta$
داریم:
\begin{equation*}
    \delta(q, a) =
      \begin{cases}
        \{0, \delta(q_L(1), a)\} & q(0) = 0, a \ne \epsilon\\
        \{1\} \times N(q(1)) & q(0) = 0, a = \epsilon\\
        \{1, \delta(q_L(1), a)\} & q(0) = 1\\
      \end{cases}
\end{equation*}
که در اینجا
$N(q)$
تابعی است که خانه‌هایی را بر می‌گرداند که در
$M_L$
می‌توانستیم از حالت
$q$
به آن رسید.
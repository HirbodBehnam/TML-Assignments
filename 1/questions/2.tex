\section{هم‌ارزی و کمینه سازی}
\subsection{}
در ابتدا باید
\lr{starting state}
این ماشین را پیدا کنیم. بدین منظور کاری که می‌کنیم این است که می‌بینیم که از حالت
1
به کمک
$\epsilon$
به کدام حالات می‌توان رسید. این حالات عبارتند از
$\{1,2,5,6\}$.
حال سعی می‌‌کنیم که از این
\lr{state}
شروع کنیم و ببینیم که به ازای
$a$
و
$b$
از هر
\lr{state}
درون هر
\lr{state}
به کدام یک از حالات می‌توان رسید.
\begin{latin}
    \centering
    \begin{tabular}{c|cc}
        & a & b\\
        \hline
        $\{1,2,5,6\}$ & $\{1,2,3,5,6\}$ & $\{5,6,7\}$ \\
        $\{1,2,3,5,6\}$ & $\{1,2,3,5,6\}$ & $\{4,5,6,7\}$ \\
        $\{5,6,7\}$ & $\{6\}$ & $\{5,6,7\}$ \\
        $\{4,5,6,7\}$ & $\{6\}$ & $\{2,5,6,7\}$ \\
        $\{6\}$ & - & $\{7\}$ \\
        $\{2,5,6,7\}$ & $\{3,6\}$ & $\{5,6,7\}$ \\
        $\{7\}$ & $\{6\}$ & - \\
        $\{3,6\}$ & - & $\{4,7\}$\\
        $\{4,7\}$ & $\{6\}$ & $\{2\}$\\
        $\{2\}$ & $\{3\}$ & - \\
        $\{3\}$ & - & $\{4\}$\\
        $\{4\}$ & - & $\{2\}$
    \end{tabular}
\end{latin}
در انتها نیز باید تعیین کنیم که کدام یک از حالات بدست آمده
\lr{accepting state}
هستند. جواب آنهایی است که در مجموعه‌ی خود 2 یا 6 را دارند. پس
\begin{gather*}
    F = \left\{\{1,2,5,6\},\{1,2,3,5,6\},\{5,6,7\},\{4,5,6,7\},\{6\},\{2,5,6,7\},\{3,6\},\{2\}\right\}
\end{gather*}
\subsection{}
در ابتدا چک می‌کنیم که آیا
\lr{dead state} و \lr{unreachable state}
در
\lr{DFA}
وجود دارد یا خیر. از آنجا که تمامی
\lr{state}ها
ورودی دارند پس
\lr{unreachable state}
نداریم. همچنین در هر
\lr{state}ای
که باشیم می‌توان به تمامی
\lr{state}های
دیگر رسید.

% https://www.youtube.com/watch?v=7W2lSrt8r-0
حال باید حالات متمایز را تشخبص دهیم. در ابتدا جدولی تشکیل می‌دهیم که در آن نگه می‌داریم که
کدام حالات نسبت به هم متمایز هستند. در ابتدا مشخص است که حالات
\lr{acceptable} و \lr{non acceptable}
متمایز هستند. پس داریم:
\begin{latin}
    \centering
    \begin{tabular}{c|c|c|c|c|c|c|}
          & 1 & 2 & 3 & 4 & 5 & 6\\
        \cline{0-6}
        7 & x & x & x & x & x &  \\
        \cline{0-6}
        6 & x & x & x & x & x\\
        \cline{0-5}
        5 &   &   &   &  \\
        \cline{0-4}
        4 &   &   &  \\
        \cline{0-3}
        3 &   &  \\
        \cline{0-2}
        2 &  \\
        \cline{0-1}
    \end{tabular}
\end{latin}
در ادامه هر دو
\lr{state}
را در نظر می‌گیریم و
$a$ و $b$
را همزمان به آن
\lr{state}ها
می‌دهیم. در صورتی که در با دادن هر کدام از آنها در حالتی رفتیم که
\lr{accepting}
بود و دگیری
\lr{non accepting}،
خانه‌ی این دو حالت را علامت دار می‌کنیم. بعد از این کار به ازای
هر دو حالت داریم:
\begin{latin}
    \centering
    \begin{tabular}{c|c|c|c|c|c|c|}
          & 1 & 2 & 3 & 4 & 5 & 6\\
        \cline{0-6}
        7 & x & x & x & x & x &  \\
        \cline{0-6}
        6 & x & x & x & x & x\\
        \cline{0-5}
        5 & x & x & x &  \\
        \cline{0-4}
        4 & x & x & x\\
        \cline{0-3}
        3 & x & x\\
        \cline{0-2}
        2 &  \\
        \cline{0-1}
    \end{tabular}
\end{latin}
حال بار دیگر برای حالت‌هایی که علامت زده شده نسیتند کار فوق
را انجام می‌دهیم و چک می‌کنیم که آیا در حالاتی می‌افتیم که
نسبت به هم متمایز باشند یا خیر. در صورتی که متمایز باشند،
برای دو حالت جدید نیز یک علامت قرار می‌دهیم.
\begin{latin}
    \centering
    \begin{tabular}{c|c|c|c|c|c|c|}
          & 1 & 2 & 3 & 4 & 5 & 6\\
        \cline{0-6}
        7 & x & x & x & x & x &  \\
        \cline{0-6}
        6 & x & x & x & x & x\\
        \cline{0-5}
        5 & x & x & x &  \\
        \cline{0-4}
        4 & x & x & x\\
        \cline{0-3}
        3 & x & x\\
        \cline{0-2}
        2 & x\\
        \cline{0-1}
    \end{tabular}
\end{latin}
حال در صورتی که بار دیگر این کار را انجام دهیم متوجه می‌شویم که
جدول تغییری نمی‌کند. پس می‌توان حالات 6 و 7 را با هم، و حالات 4 و 5 را نیز با هم ترکیب کرد.
\lr{DFA}
نهایی به صورت زیر است:
\begin{latin}
    \centering
    \begin{tikzpicture}[->, >=stealth', auto, semithick, node distance=3cm]
    \tikzstyle{every state}=[fill=white,draw=black,thick,text=black,scale=1]
    \node[initial,state,initial text=] (1)  {$1$};
    \node[state,accepting] (67)[right of=1] {$6,7$};
    \node[state] (2)[above of=67]  {$2$};
    \node[state] (45)[below of=67] {$4,5$};
    \node[state] (3)[right of=67]  {$3$};
    \path
    (1) edge[below] node{$b$} (67)
    (1) edge[bend right, below] node{$a$} (45)
    (2) edge[bend right, above] node{$a$} (1)
    (2) edge[bend right, left] node{$b$} (67)
    (3) edge[bend right, above] node{$a$} (2)
    (3) edge[bend left, below] node{$b$} (45)
    (45) edge[right] node{$a$} (67)
    (45) edge[loop below] node{$b$} (45)
    (67) edge[below] node{$a$} (3)
    (67) edge[out=60,in=30,looseness=8] node{$b$} (67)
    ;
    \end{tikzpicture}
\end{latin}
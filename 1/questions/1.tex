\section{مفاهیم ماشین‌های حالت متناهی}
\subsection{}
\begin{enumerate}
    \item برای این قسمت 16 \lr{state} مختلف تعریف می‌کنیم.
    شماره هر \lr{state}
    در مبنای دو نشان‌دهنده‌ی این است که کدام یک از حروف را در رشته دیده‌ایم.
    به عنوان مثال حالت
    1101
    نشان می‌دهد که فقط
    $q$
    را ندیده‌ایم. با این توضیحات مشخص است که از حالت 0 باید شروع کنیم
    (که هیچ حرفی را ندیدیم)
    و همه‌ی حالات
    \lr{accepting state}
    هستند غیر از حالت شماره
    15
    که در آن تمامی حروف را دیده‌ایم. اما این
    \lr{DFA}
    رسمش بسیار پیچیده می شود! کار دیگری که می توان کرد که تعداد
    \lr{state}ها
    را بیشتر می کند این است که ساختاری شبیه درخت داشته باشیم.
    به شکل زیر توجه کنید:
    % Used https://madebyevan.com/fsm/
    \begin{latin}
        \begin{center}
            \begin{tikzpicture}[->, >=stealth', auto, semithick, node distance=2cm]
            \tikzstyle{every state}=[fill=white,draw=black,thick,text=black,scale=1]
            \node[initial,state,initial text=]    (Q0)         {$q_0$};
            % First ring
            \node[state]    (Q1)[right of=Q0]    {$q_1$};
            \node[state]    (Q2)[below of=Q0]    {$q_2$};
            \node[state]    (Q3)[left of=Q0]     {$q_3$};
            \node[state]    (Q4)[above of=Q0]    {$q_4$};
            % Second ring
            \node[state]    (Q5)[above right of=Q1] {$q_5$};
            \node[state]    (Q6)[right of=Q1]       {$q_6$};
            \node[state]    (Q7)[below right of=Q1] {$q_7$};
            \node[state]    (Q8)[below right of=Q2] {$q_8$};
            \node[state]    (Q9)[below of=Q2]       {$q_9$};
            \node[state]    (Q10)[below left of=Q2] {$q_{10}$};
            \node[state]    (Q11)[below left of=Q3] {$q_{11}$};
            \node[state]    (Q12)[left of=Q3]       {$q_{12}$};
            \node[state]    (Q13)[above left of=Q3] {$q_{13}$};
            \node[state]    (Q14)[above left of=Q4] {$q_{14}$};
            \node[state]    (Q15)[above of=Q4]      {$q_{15}$};
            \node[state]    (Q16)[above right of=Q4] {$q_{16}$};
            % Third ring
            \node[state]    (Q17)[above right of=Q5] {$q_{17}$};
            \node[state]    (Q18)[right of=Q5] {$q_{18}$};
            \node[state]    (Q19)[above right of=Q6] {$q_{19}$};
            \node[state]    (Q20)[below right of=Q6] {$q_{20}$};
            \path
            % First letter
            (Q0) edge[above] node{$p$} (Q1)
            (Q0) edge[left] node{$q$} (Q2)
            (Q0) edge[bend right, above] node{$r$} (Q3)
            (Q0) edge[left] node{$s$} (Q4)
            % Second letters
            (Q1) edge[above] node{$q$} (Q5)
            (Q1) edge[above] node{$r$} (Q6)
            (Q1) edge[above] node{$s$} (Q7)
            (Q2) edge[left] node{$p$} (Q8)
            (Q2) edge[left] node{$r$} (Q9)
            (Q2) edge[left] node{$s$} (Q10)
            (Q3) edge[above] node{$p$} (Q11)
            (Q3) edge[above] node{$q$} (Q12)
            (Q3) edge[above] node{$s$} (Q13)
            (Q4) edge[left] node{$p$} (Q14)
            (Q4) edge[left] node{$q$} (Q15)
            (Q4) edge[left] node{$r$} (Q16);
            \end{tikzpicture}
        \end{center}
    \end{latin}
\end{enumerate}







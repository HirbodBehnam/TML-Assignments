\section{مفاهیم ماشین‌های حالت متناهی}
\subsection{}
\begin{enumerate}
    \item برای این قسمت 16 \lr{state} مختلف تعریف می‌کنیم.
    شماره هر \lr{state}
    در مبنای دو نشان‌دهنده‌ی این است که کدام یک از حروف را در رشته دیده‌ایم.
    به عنوان مثال حالت
    1101
    نشان می‌دهد که فقط
    $q$
    را ندیده‌ایم. با این توضیحات مشخص است که از حالت 0 باید شروع کنیم
    (که هیچ حرفی را ندیدیم)
    و همه‌ی حالات
    \lr{accepting state}
    هستند غیر از حالت شماره
    15
    که در آن تمامی حروف را دیده‌ایم. اما این
    \lr{DFA}
    رسمش بسیار پیچیده می شود! برای همین صرفا آنرا تعریف می‌کنیم:
    \begin{align*}
        M &= (Q, \Sigma, \delta, q_0, F)\\
        Q &= \{q_0,q_1,\dots,q_{15}\}\\
        F &= \{q_0,q_1,\dots,q_{14}\}
    \end{align*}
    حال عضو‌های مجموعه‌ی
    $\delta$
    را می‌نویسیم:
    \[
    \begin{array}{cccc}
        % Start
        (q_0, p) \rightarrow q_1 & (q_0, q) \rightarrow q_2 & (q_0, r) \rightarrow q_4 & (q_0, s) \rightarrow q_8\\
        % First letters
        (q_1, p) \rightarrow q_1 & (q_1, q) \rightarrow q_3 & (q_1, r) \rightarrow q_5 & (q_1, s) \rightarrow q_9\\
        (q_2, p) \rightarrow q_3 & (q_2, q) \rightarrow q_2 & (q_2, r) \rightarrow q_6 & (q_2, s) \rightarrow q_{10}\\
        (q_4, p) \rightarrow q_5 & (q_4, q) \rightarrow q_6 & (q_4, r) \rightarrow q_4 & (q_4, s) \rightarrow q_{12}\\
        (q_8, p) \rightarrow q_9 & (q_8, q) \rightarrow q_{10} & (q_8, r) \rightarrow q_{12} & (q_8, s) \rightarrow q_{8}\\
        % Second letters
        (q_3, p) \rightarrow q_{3} & (q_3, q) \rightarrow q_{3} & (q_3, r) \rightarrow q_{7} & (q_3, s) \rightarrow q_{11}\\
        (q_5, p) \rightarrow q_{5} & (q_5, q) \rightarrow q_{7} & (q_5, r) \rightarrow q_{5} & (q_5, s) \rightarrow q_{13}\\
        (q_6, p) \rightarrow q_{7} & (q_6, q) \rightarrow q_{6} & (q_6, r) \rightarrow q_{6} & (q_6, s) \rightarrow q_{14}\\
        (q_9, p) \rightarrow q_{9} & (q_9, q) \rightarrow q_{11} & (q_9, r) \rightarrow q_{13} & (q_9, s) \rightarrow q_{9}\\
        (q_{10}, p) \rightarrow q_{11} & (q_{10}, q) \rightarrow q_{10} & (q_{10}, r) \rightarrow q_{14} & (q_{10}, s) \rightarrow q_{10}\\
        (q_{12}, p) \rightarrow q_{13} & (q_{12}, q) \rightarrow q_{14} & (q_{12}, r) \rightarrow q_{12} & (q_{12}, s) \rightarrow q_{12}\\
        % Third letters
        (q_{7}, p) \rightarrow q_{7} & (q_{7}, q) \rightarrow q_{7} & (q_{7}, r) \rightarrow q_{7} & (q_{7}, s) \rightarrow q_{15}\\
        (q_{11}, p) \rightarrow q_{11} & (q_{11}, q) \rightarrow q_{11} & (q_{11}, r) \rightarrow q_{15} & (q_{11}, s) \rightarrow q_{11}\\
        (q_{13}, p) \rightarrow q_{13} & (q_{13}, q) \rightarrow q_{15} & (q_{13}, r) \rightarrow q_{13} & (q_{13}, s) \rightarrow q_{13}\\
        (q_{14}, p) \rightarrow q_{15} & (q_{14}, q) \rightarrow q_{14} & (q_{14}, r) \rightarrow q_{14} & (q_{14}, s) \rightarrow q_{14}\\
    \end{array}
    \]
    \item در این قسمت عملا یک جور مموری نگه می‌داریم از 4 کارکتر قبلی که خوانده بودیم. سپس با خواندن یک
    کارکتر جدید عملا باید اولین حرفی که خوانده بودیم را دور بریزیم و وارد حالتی شویم که سه حرفی که خوانده
    بودیم برابر 4 حرف آخر ما هستند. به عنوان مثال رشته‌ی
    $baaca$
    را در نظر بگیرید. در ابتدا با خواندن 4 حرف اول وارد حالت
    $q_{baac}$
    می‌شویم. سپس با خواندن
    $a$
    وارد حالت
    $aaca$
    می‌شویم. همچنین دقت کنید که در برخی از مواقع ماشین ما
    \lr{die off}
    می‌شود. فرض کنید که تا به حال
    $aabb$
    را خوانده‌ایم و یک
    $b$
    دیگر می‌آید. در این صورت می‌توان یا به یک حالت
    \lr{trap}
    رفت یا اینکه به کل این
    \lr{trasition}
    را قرار ندهیم.
    قسمتی از این ماشین به صورت زیر است:
    \begin{latin}
        \centering
        \begin{tikzpicture}[->, auto, semithick, node distance=2.5cm]
        \tikzstyle{every state}=[fill=white,draw=black,thick,text=black,scale=1]
        \node[initial,state,accepting,initial text=] (0)  {$q_{\epsilon}$};
        \node[state,accepting] (a)[above right of=0] {$q_{a}$};
        \node[state,accepting] (aa)[above right of=a]  {$q_{aa}$};
        \node[state,accepting] (aaa)[above right of=aa] {$q_{aaa}$};
        \node[state,accepting] (aaaa)[above right of=aaa]  {$q_{aaaa}$};
        \node[state,accepting] (aaab)[right of=aaa]  {$q_{aaab}$};
        \node[state,accepting] (aaac)[above of=aaa]  {$q_{aaac}$};
        \node[state,accepting] (aab)[right of=aa]  {$q_{aab}$};
        \node[state,accepting] (aaba)[right of=aab]  {$q_{aaba}$};
        \node[state,accepting] (aabb)[below right of=aab]  {$q_{aabb}$};
        \path
        (0) edge[above] node{$a$} (a)
        (a) edge[above] node{$a$} (aa)
        (aa) edge[above] node{$a$} (aaa)
        (aaa) edge[above] node{$a$} (aaaa)
        (aaa) edge[above] node{$b$} (aaab)
        (aaa) edge[right] node{$c$} (aaac)
        (aaaa) edge[right] node{$b$} (aaab)
        (aaaa) edge[loop above] node{$a$} (aaaa)

        (aa) edge[above] node{$b$} (aab)
        (aab) edge[above] node{$a$} (aaba)
        (aab) edge[above] node{$b$} (aabb)
        (aaab) edge[left] node{$a$} (aaba)
        ;
        \draw[out=-20, in=20, looseness=1.7] (aaab) to node[pos=0.5]{$b$} (aabb) ;
        \end{tikzpicture}
    \end{latin}
    در شکل بالا در صورتی که در حالت
    $q_{aabb}$
    باشیم و
    $b$
    را بخوانیم یا باید در
    \lr{trap state}
    برویم یا اینکه با عدم آوردن این
    \lr{trasition}
    در
    $\delta$
    نشان دهیم که ماشین این رشته را قبول نمی‌کند.
    \begin{latin}
    \centering
    \begin{longtable}{c|ccc}
        \backslashbox{State}{Input} & $a$ & $b$ & $c$\\
        \hline
        $q_{aaaa}$ & $q_{aaaa}$ & $q_{aaab}$ & $q_{aaac}$ \\
$q_{aaab}$ & $q_{aaba}$ & $q_{aabb}$ & $q_{aabc}$ \\
$q_{aaac}$ & $q_{aaca}$ & $q_{aacb}$ & $q_{aacc}$ \\
$q_{aaba}$ & $q_{abaa}$ & $q_{abab}$ & $q_{abac}$ \\
$q_{aabb}$ &  &  &  \\
$q_{aabc}$ &  &  & $q_{abcc}$ \\
$q_{aaca}$ & $q_{acaa}$ & $q_{acab}$ & $q_{acac}$ \\
$q_{aacb}$ &  &  & $q_{acbc}$ \\
$q_{aacc}$ & $q_{acca}$ & $q_{accb}$ & $q_{accc}$ \\
$q_{abaa}$ & $q_{baaa}$ & $q_{baab}$ & $q_{baac}$ \\
$q_{abab}$ & $q_{baba}$ &  &  \\
$q_{abac}$ & $q_{baca}$ &  & $q_{bacc}$ \\
$q_{abcc}$ & $q_{bcca}$ & $q_{bccb}$ & $q_{bccc}$ \\
$q_{acaa}$ & $q_{caaa}$ & $q_{caab}$ & $q_{caac}$ \\
$q_{acab}$ & $q_{caba}$ &  &  \\
$q_{acac}$ & $q_{caca}$ & $q_{cacb}$ & $q_{cacc}$ \\
$q_{acbc}$ & $q_{cbca}$ & $q_{cbcb}$ & $q_{cbcc}$ \\
$q_{acca}$ & $q_{ccaa}$ & $q_{ccab}$ & $q_{ccac}$ \\
$q_{accb}$ & $q_{ccba}$ & $q_{ccbb}$ & $q_{ccbc}$ \\
$q_{accc}$ & $q_{ccca}$ & $q_{cccb}$ & $q_{cccc}$ \\
$q_{baaa}$ & $q_{aaaa}$ & $q_{aaab}$ & $q_{aaac}$ \\
$q_{baab}$ & $q_{aaba}$ & $q_{aabb}$ & $q_{aabc}$ \\
$q_{baac}$ & $q_{aaca}$ & $q_{aacb}$ & $q_{aacc}$ \\
$q_{baba}$ & $q_{abaa}$ & $q_{abab}$ & $q_{abac}$ \\
$q_{baca}$ & $q_{acaa}$ & $q_{acab}$ & $q_{acac}$ \\
$q_{bacc}$ & $q_{acca}$ & $q_{accb}$ & $q_{accc}$ \\
$q_{bbaa}$ & $q_{baaa}$ & $q_{baab}$ & $q_{baac}$ \\
$q_{bbcc}$ & $q_{bcca}$ & $q_{bccb}$ & $q_{bccc}$ \\
$q_{bcaa}$ & $q_{caaa}$ & $q_{caab}$ & $q_{caac}$ \\
$q_{bcac}$ & $q_{caca}$ & $q_{cacb}$ & $q_{cacc}$ \\
$q_{bcbc}$ & $q_{cbca}$ & $q_{cbcb}$ & $q_{cbcc}$ \\
$q_{bcca}$ & $q_{ccaa}$ & $q_{ccab}$ & $q_{ccac}$ \\
$q_{bccb}$ & $q_{ccba}$ & $q_{ccbb}$ & $q_{ccbc}$ \\
$q_{bccc}$ & $q_{ccca}$ & $q_{cccb}$ & $q_{cccc}$ \\
$q_{caaa}$ & $q_{aaaa}$ & $q_{aaab}$ & $q_{aaac}$ \\
$q_{caab}$ & $q_{aaba}$ & $q_{aabb}$ & $q_{aabc}$ \\
$q_{caac}$ & $q_{aaca}$ & $q_{aacb}$ & $q_{aacc}$ \\
$q_{caba}$ & $q_{abaa}$ & $q_{abab}$ & $q_{abac}$ \\
$q_{caca}$ & $q_{acaa}$ & $q_{acab}$ & $q_{acac}$ \\
$q_{cacb}$ &  &  & $q_{acbc}$ \\
$q_{cacc}$ & $q_{acca}$ & $q_{accb}$ & $q_{accc}$ \\
$q_{cbaa}$ & $q_{baaa}$ & $q_{baab}$ & $q_{baac}$ \\
$q_{cbca}$ & $q_{bcaa}$ &  & $q_{bcac}$ \\
$q_{cbcb}$ &  &  & $q_{bcbc}$ \\
$q_{cbcc}$ & $q_{bcca}$ & $q_{bccb}$ & $q_{bccc}$ \\
$q_{ccaa}$ & $q_{caaa}$ & $q_{caab}$ & $q_{caac}$ \\
$q_{ccab}$ & $q_{caba}$ &  &  \\
$q_{ccac}$ & $q_{caca}$ & $q_{cacb}$ & $q_{cacc}$ \\
$q_{ccba}$ & $q_{cbaa}$ &  &  \\
$q_{ccbb}$ &  &  &  \\
$q_{ccbc}$ & $q_{cbca}$ & $q_{cbcb}$ & $q_{cbcc}$ \\
$q_{ccca}$ & $q_{ccaa}$ & $q_{ccab}$ & $q_{ccac}$ \\
$q_{cccb}$ & $q_{ccba}$ & $q_{ccbb}$ & $q_{ccbc}$ \\
$q_{cccc}$ & $q_{ccca}$ & $q_{cccb}$ & $q_{cccc}$ \\
    \end{longtable}
    \end{latin}
    دقت کنید که علاوه بر حالات گفته شده، حالات میانی نیز وجود دارند که از آنها صرفا به حالاتی می‌رویم که یک
    حرف بیشتر دارند به عنوان مثال:
    \begin{latin}
        \centering
        \begin{longtable}{c|ccc}
            \backslashbox{State}{Input} & $a$ & $b$ & $c$\\
            \hline
            $q_{ab}$ & $q_{aba}$ & $q_{abb}$ & $q_{abc}$ 
        \end{longtable}
    \end{latin}
    پس کل ماشین ما
    $1 + 3 + 9 + 27 + 54$
    حالت دارد.
    همچنین تمامی حالات این ماشین
    \lr{accepting}
    هستند.
\end{enumerate}
\subsection{}
به صورت شهودی سوال از ما می‌خواهد اثبات کنیم ماشین
$A$
معادل
\lr{DFA}
ماشین غیر قطعی
$B$
است. برای شروع اثبات، استقرا را بر روی طول رشته می‌زنیم. برای پایه‌ی استقرا فرض می‌کنیم که طول رشته‌ی
$w$
برابر 1 است. در این صورت داریم:
(فرض می‌کنیم که $a$ یک حرف در زبان است.)
\begin{gather*}
    \hat{\delta}^*(q_0, a) = \hat{\delta}(\hat{\delta}^*(q_0, \epsilon), a) = \hat{\delta}(\{\hat{q_0}\}, a)\\
    \delta^*(\{\hat{q_0}\}, a) = \delta(\delta^*(q_0, \epsilon), a) = \delta(q_0, a)
\end{gather*}
حال از فرض خود سوال داریم:
\begin{gather*}
    \delta(q_0, a) = \bigcup_{\hat{s} \in q_0} \hat{\delta}(\hat{s}, a) =
    \bigcup_{\hat{s} \in \{\hat{q_0}\}} \hat{\delta}(\hat{s}, a) = \hat{\delta}(\{\hat{q_0}\}, a)
\end{gather*}
حال خود استقرا را می‌زنیم. فرض کنید که فرض استقرا
$\hat{\delta}^*(q_0, w) = \delta^*(\{\hat{q_0}\}, w)$
است که
$w$
یک رشته است. می‌خواهیم ثابت کنیم:
\begin{gather*}
    \hat{\delta}^*(q_0, wa) = \delta^*(\{\hat{q_0}\}, wa)
\end{gather*}
طبق تعاریف داریم:
\begin{gather*}
    \hat{\delta}^*(q_0, wa) = \hat{\delta}(\hat{\delta}^*(q_0, w), a) \stackrel{\text{فرض استقرا و سوال}}{=}
    \delta(\delta^*(\{\hat{q_0}\}, w), a) = \delta^*(\{\hat{q_0}\}, wa)
\end{gather*}
\textbf{پی‌نوشت:}
می‌توانستیم پایه‌ی استقرا را نیز کمی بهتر و اساسی‌تر بگیریم و بر روی اپسیلون استقرا بزنیم.
ولی استقرا زدن بر روی یک رشته‌ی یک حرفی حداقل به من کمک کرد که گام استقرا را متوجه شوم.
در صورت نیاز پایه‌ی استقرا با
$\epsilon$
به صورت زیر است:
\begin{gather*}
    \hat{\delta}^*(q_0, \epsilon) = q_0\\
    \delta^*(\{\hat{q_0}\}, \epsilon) = \{\hat{q_0}\} = q_0
\end{gather*}
\subsection{}
زبان
$L' = b^*a$
را در نظر بگیرید که زیر مجموعه‌ی
$L$
است.
\lr{NFA}
این زبان به صورت زیر است:
\begin{latin}
    \centering
    \begin{tikzpicture}[->, >=stealth', auto, semithick, node distance=3cm]
    \tikzstyle{every state}=[fill=white,draw=black,thick,text=black,scale=1]
    \node[initial,state,initial text=] (1)  {$1$};
    \node[state,accepting] (2)[right of=1] {$2$};
    \path
    (1) edge[loop above] node{$b$} (1)
    (1) edge[below] node{$a$} (2)
    ;
    \end{tikzpicture}
\end{latin}
در صورتی که مثل صورت سوال برای تبدیل این ماشین به یک
\lr{NFA}
که
$a^*b^*a$
را می‌پذیرد باید ماشین را تبدیل به ماشین زیر کنیم:
\begin{latin}
    \centering
    \begin{tikzpicture}[->, >=stealth', auto, semithick, node distance=3cm]
    \tikzstyle{every state}=[fill=white,draw=black,thick,text=black,scale=1]
    \node[initial,state,initial text=] (1)  {$1$};
    \node[state,accepting] (2)[right of=1] {$2$};
    \path
    (1) edge[loop above] node{$b$} (1)
    (1) edge[loop below] node{$a$} (1)
    (1) edge[below] node{$a$} (2)
    ;
    \end{tikzpicture}
\end{latin}
اما این ماشین درست نیست چرا که
$baba$
را می‌پذیرد ولی این در زبان ما نیست.


برای حالت دوم هم مثالی مشابه را در نظر میگیریم:
$L' = ab^*$
\begin{latin}
    \centering
    \begin{tikzpicture}[->, >=stealth', auto, semithick, node distance=3cm]
    \tikzstyle{every state}=[fill=white,draw=black,thick,text=black,scale=1]
    \node[initial,state,initial text=] (1)  {$1$};
    \node[state,accepting] (2)[right of=1] {$2$};
    \path
    (1) edge[below] node{$a$} (2)
    (2) edge[loop above] node{$b$} (2)
    ;
    \end{tikzpicture}
\end{latin}
حال زبان
$ab^*a^*$
را به صورت گفته شده در صورت سوال می‌نویسیم:
\begin{latin}
    \centering
    \begin{tikzpicture}[->, >=stealth', auto, semithick, node distance=3cm]
    \tikzstyle{every state}=[fill=white,draw=black,thick,text=black,scale=1]
    \node[initial,state,initial text=] (1)  {$1$};
    \node[state,accepting] (2)[right of=1] {$2$};
    \path
    (1) edge[below] node{$a$} (2)
    (2) edge[loop above] node{$b$} (2)
    (2) edge[loop below] node{$a$} (2)
    ;
    \end{tikzpicture}
\end{latin}
این
\lr{NFA}
رشته‌ی
$ababa$
را قبول می‌کند ولی این رشته در زبان ما نیست!

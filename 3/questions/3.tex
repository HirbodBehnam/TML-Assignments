\section{ماشین‌های پشته‌ای}
\subsection{}
\begin{enumerate}
    \item این ماشین زبان‌های متقارن با طول زوج را نمی‌پذیرد. به عنوان یک مثال ساده
    $11$
    را نمی‌پذیرد.
    \item کافی است که یک \lr{epsilon transition} یا به عبارتی
    $\epsilon, \epsilon \rightarrow \epsilon$
    از
    $q_1$ به $q_2$
    وصل کنیم.
\end{enumerate}
\subsection{}
\begin{enumerate}
    \item صرفا در زمانی که اولین $b$ را می‌خوانیم از
    \lr{non determinism}
    استفاده می‌کنیم.
    \begin{latin}
        \centering
        \begin{tikzpicture}[->, >=stealth', auto, semithick, node distance=3cm]
        \tikzstyle{every state}=[fill=white,draw=black,thick,text=black,scale=1]
        \node[initial,state,initial text=] (0)  {$q_0$};
        \node[state] (1)[right of=0]  {$q_1$};
        \node[state] (2)[above right of=1] {$q_2$};
        \node[state] (3)[below right of=1] {$q_3$};
        \node[state] (4)[below of=3] {$q_4$};
        \node[state] (5)[below of=4] {$q_5$};
        \node[state,accepting] (6)[below right of=2] {$q_6$};
        \path
        (0) edge[above] node{$\epsilon, \epsilon \rightarrow \$$} (1)
        (1) edge[loop below] node{$a, \epsilon \rightarrow a$} (1)
        (2) edge[loop above] node{$b, a \rightarrow \epsilon$} (2)
        (1) edge[sloped, anchor=center, above] node{$b, a \rightarrow \epsilon$} (2)
        (1) edge[sloped, anchor=center, above] node{$b, a \rightarrow \epsilon$} (3)
        (3) edge[sloped, anchor=center, below] node{$b, a \rightarrow \epsilon$} (4)
        (4) edge[sloped, anchor=center, below] node{$b, a \rightarrow \epsilon$} (5)
        (2) edge[sloped, anchor=center, above] node{$\epsilon, \$ \rightarrow \epsilon$} (6)
        (5) edge[bend right, right] node{$\epsilon, \$ \rightarrow \epsilon$} (6)
        (5) edge[bend left, left] node{$b, a \rightarrow \epsilon$} (3)
        ;
        \end{tikzpicture}
    \end{latin}
    \item با کمی دقت متوجه می‌شویم که این زبان عملا همان
    $\{a^i b^j c^k | k \ge i ~\text{or}~ k \ge j\}$
    است. پس می‌توانیم از
    \lr{non determinism}
    استفاده کنیم.
    \begin{latin}
        \centering
        \begin{tikzpicture}[->, >=stealth', auto, semithick, node distance=3cm]
        \tikzstyle{every state}=[fill=white,draw=black,thick,text=black,scale=1]
        \node[initial,state,initial text=] (0)  {$q_0$};
        \node[state] (1)[right of=0]  {$q_1$};
        \node[state] (2)[above right of=1] {$q_2$};
        \node[state] (3)[below right of=1] {$q_3$};
        \node[state] (4)[right of=2] {$q_4$};
        \node[state] (5)[right of=3] {$q_5$};
        \node[state] (6)[below right of=4] {$q_6$};
        \node[state,accepting] (7)[right of=6] {$q_7$};
        \path
        (0) edge[above] node{$\epsilon, \epsilon \rightarrow \$$} (1)
        (1) edge[sloped, anchor=center, above] node{$a, \epsilon \rightarrow x$} (2)
        (1) edge[sloped, anchor=center, above] node{$a, \epsilon \rightarrow \epsilon$} (3)
        (2) edge[loop above] node{$a, \epsilon \rightarrow x$} (2)
        (3) edge[loop below] node{$a, \epsilon \rightarrow \epsilon$} (3)
        (2) edge[above] node{$b, \epsilon \rightarrow \epsilon$} (4)
        (3) edge[below] node{$b, \epsilon \rightarrow x$} (5)
        (4) edge[loop above] node{$b, \epsilon \rightarrow \epsilon$} (4)
        (5) edge[loop below] node{$b, \epsilon \rightarrow x$} (5)
        (4) edge[sloped, anchor=center, below] node{$c, x \rightarrow \epsilon$} (6)
        (5) edge[sloped, anchor=center, below] node{$c, x \rightarrow \epsilon$} (6)
        (6) edge[loop above] node{$c, x \rightarrow \epsilon$} (6)
        (6) edge[sloped, anchor=center, above] node{$c, \$ \rightarrow \epsilon$} (7)
        (7) edge[loop below] node{$c, \epsilon \rightarrow \epsilon$} (7)
        ;
        \end{tikzpicture}
    \end{latin}
    اما نکته‌ای که وجود دارد این است که این ماشین رشته‌هایی مثل
    $\epsilon$ یا $ac$ را قبول نمی‌کند.
    برای درست شدن این موضوع می‌توان
    \lr{transition}هایی
    مثلا از
    $q_1$
    به
    $q_5$
    با عبارت روی یال
    $b, \epsilon \rightarrow x$
    اضافه کرد. یا مثال یال
    $q_0$ به $q_7$ با یال
    $\epsilon, \epsilon \rightarrow \epsilon$
    وصل می‌کنیم.
    \item دقت کنید که شرط
    $n \neq 2m + 1$
    برابر است با
    $n - 1 \neq 2m$.
    پس اولین
    $0$
    را که می‌خوانیم هیچ چیزی را در
    \lr{stack push}
    نمی‌کنیم ولی از بعد از آن به ازای هر
    $0$
    یک
    $0$
    نیز در
    \lr{stack push}
    می‌کنیم. سپس به ازای هر
    $1$
    دو
    $0$
    را از
    \lr{stack pop}
    می‌کنیم. در صورتی که در آخر کار استک خالی بود
    \lr{reject}
    می‌کنیم و در غیر این صورت
    \lr{accept}.
    \begin{latin}
        \centering
        \begin{tikzpicture}[->, >=stealth', auto, semithick, node distance=3cm]
        \tikzstyle{every state}=[fill=white,draw=black,thick,text=black,scale=1]
        \node[initial,state,initial text=] (0)  {$q_0$};
        \node[state] (1)[below of=0]  {$q_1$};
        \node[accepting,state] (2)[below of=1] {$q_2$};
        \node[state] (3)[right of=1] {$q_3$};
        \node[state] (4)[right of=3] {$q_4$};
        \node[accepting,state] (5)[below of=4] {$q_5$};
        \path
        (0) edge[sloped, anchor=center, below] node{$0, \epsilon \rightarrow \$$} (1)
        (1) edge[loop left] node{$0, \epsilon \rightarrow 0$} (2)
        (1) edge[sloped, anchor=center, below] node{$1, \$ \rightarrow \epsilon$} (2)
        (2) edge[loop left] node{$1, \epsilon \rightarrow \epsilon$} (2)
        (1) edge[sloped, anchor=center, below] node{$1, 0 \rightarrow \epsilon$} (3)
        (3) edge[sloped, anchor=center, below] node{$\epsilon, \$ \rightarrow \epsilon$} (2)
        (3) edge[sloped, anchor=center, below] node{$\epsilon, 0 \rightarrow \epsilon$} (4)
        (4) edge[sloped, anchor=center, below] node{$1, \$ \rightarrow \epsilon$} (2)
        (4) edge[sloped, anchor=center, below] node{$\epsilon, 0 \rightarrow \epsilon$} (5)
        (4) edge[bend right, above] node{$1, 0 \rightarrow \epsilon$} (3)
        ;
        \end{tikzpicture}
    \end{latin}
\end{enumerate}
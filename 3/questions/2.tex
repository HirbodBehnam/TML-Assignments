\section{خواص بستاری زبان‌های مستقل از متن}
\subsection{}
\begin{enumerate}
    \item فرض کنید که گرامرهای
    $G = (V, \Sigma, R, S)$ و $G' = (V', \Sigma', R', S')$
    را داریم. گرامر
    $G''$
    را به صورت زیر تشکیل می‌دهم:
    \begin{align*}
        V'' &= V \cup V'\\
        \Sigma'' &= \Sigma \cup \Sigma'\\
        R'' &= R \cup R' \cup \{S'' \rightarrow SS'\}
    \end{align*}
    \item فرض کنید که گرامر ما
    $G = (V, \Sigma, R, S)$
    است. در این حالت گرامر
    $G' = (V, \Sigma, R', S')$
    را به صورت زیر تعریف می‌کنیم.
    \begin{align*}
        R' = R \cup \{S' \rightarrow SS' ~|~ \epsilon\}
    \end{align*}
\end{enumerate}
\subsection{}
\begin{itemize}
    \item در ابتدا اثبات می‌کنیم که اجتماع دو
    \lr{CFG}
    باز هم یک
    \lr{CFG}
    است. فرض کنید که داریم:
    $G = (V, \Sigma, R, S)$ و $G' = (V', \Sigma', R', S')$.
    کافی است که داشته باشیم:
    \begin{align*}
        V'' &= V \cup V'\\
        \Sigma'' &= \Sigma \cup \Sigma'\\
        R'' &= R \cup R' \cup \{S'' \rightarrow S ~|~ S'\}
    \end{align*}.
    با این حالت کافی است که برای هر کدام از شرط‌ها یک
    \lr{grammer}
    بنویسیم. به عنوان مثال برای
    $\{a^mb^kc^n | m,n > 0, m = k + n\}$
    داریم:
    \begin{align*}
        S &\rightarrow aAc\\
        A &\rightarrow aAc ~|~ aBb ~|~ \epsilon\\
        B &\rightarrow aBb ~|~ \epsilon
    \end{align*}
    برای قسمت
    $\{a^mb^kc^n | m,n > 0, k = m + n\}$
    داریم:
    \begin{align*}
        S &\rightarrow aAbbBc\\
        A &\rightarrow aAb ~|~ \epsilon\\
        B &\rightarrow bBc ~|~ \epsilon
    \end{align*}
    برای قسمت
    $\{a^mb^kc^n | m,n > 0, n = k + m\}$
    نیز داریم:
    \begin{align*}
        S &\rightarrow aAc\\
        A &\rightarrow aAc ~|~ bBc ~|~ \epsilon\\
        B &\rightarrow bBc ~|~ \epsilon
    \end{align*}
\end{itemize}